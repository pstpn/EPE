\chapter{Ход работы}

\section{Тренировочное задание}

\subsection{Условие задания}

Осуществить планирование проекта со следующими временными характеристиками (вариант №3):
\begin{table}[h]
	\centering
	\begin{tabularx}{\textwidth}{|X|X|}
		\hline
		\textit{Название работы} & \textit{Длительность (дни)} \\ \hline
		Работа A        & 3                 \\ \hline
		Работа B        & 4                  \\ \hline
		Работа C        & 1                 \\ \hline
		Работа D        & 4                  \\ \hline
		Работа E        & 5                  \\ \hline
		Работа F        & 7                  \\ \hline
		Работа G        & 6                 \\ \hline
		Работа H        & 5                 \\ \hline
		Работа I        & 8                  \\ \hline
	\end{tabularx}
\end{table}

Дата начала проекта~---~1-й рабочий день февраля текущего года (03.02.2025). 
Провести планирование работ проекта, учитывая следующие связи между задачами:
\begin{enumerate}
	\item предусмотреть, что D~---~исходная работа проекта;
	\item работа E следует за D;
	\item работы A, G и C следуют за E;
	\item работа B следует за A;
	\item работа H следует за G;
	\item работа F следует за C;
	\item работа I начинается после завершения B, H, и F.
 \end{enumerate}

Тип задач по умолчанию~---~\textit{Фиксированный объем ресурсов}.
Так как чем больше привлекается ресурсов, тем больше вероятность ускорить выполнение.

\subsection{Результаты выполнения задачи}

\includeimage
	{test}
	{f}
	{H}
	{\textwidth}
	{Результаты выполнения задачи}

\section{Основное задание}

\subsection{Содержание проекта}

Команда разработчиков из 16 человек занимается созданием карты города на основе собственного модуля отображения. 
Проект должен быть завершен в течение 6 месяцев. 
Бюджет проекта: 50 000 рублей.

\subsection{Задание 1: Настройка рабочей среды проекта}

На рисунке~\ref{img:task_1_1} представлены сведения о проекте.
\includeimage
	{task_1_1}
	{f}
	{H}
	{\textwidth}
	{Сведения о проекте}

\clearpage
На рисунке~\ref{img:task_1_2} представлены изменения в расписании проекта.
\includeimage
	{task_1_2}
	{f}
	{H}
	{\textwidth}
	{Изменения в расписании проекта}

На рисунке~\ref{img:task_1_3} представлено изменение настройки календаря.
\includeimage
	{task_1_3}
	{f}
	{H}
	{\textwidth}
	{Изменение настройки календаря}

\clearpage
На рисунке~\ref{img:task_1_4} представлено добавление праздников и сокращенных дней с выходными.
\includeimage
	{task_1_4}
	{f}
	{H}
	{\textwidth}
	{Добавление праздников и сокращенных дней с выходными}

\clearpage
На рисунке~\ref{img:task_1_5} представлены сведения о суммарной задаче и заметка.
\includeimage
	{task_1_5}
	{f}
	{H}
	{\textwidth}
	{Сведения о суммарной задаче и заметка}

\subsection{Задание 2: Создание списка задач}

На рисунке~\ref{img:task2} представлен список задач в проекте.
\includeimage
	{task2}
	{f}
	{H}
	{\textwidth}
	{Список задач в проекте}

\clearpage
\subsection{Задание 3: Структурирование списка задач}

На рисунке~\ref{img:task3} представлен структурированный список задач в проекте.
\includeimage
	{task3}
	{f}
	{H}
	{\textwidth}
	{Структурированный структурированный список задач}

\subsection{Задание 4: Установление связей между задачами}

На рисунке~\ref{img:task4} представлен структурированный список задач с установленными связями в проекте.
\includeimage
	{task4}
	{f}
	{H}
	{\textwidth}
	{Структурированный структурированный список задачс установленными связями}
