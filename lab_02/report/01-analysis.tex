\chapter{Ход работы}

\section{Тренировочное задание}

\subsection{Условие задания}

Вариант задания~---~3.
Дополнить временной план проекта, подготовленный на предыдущем этапе (лабораторная работа № 1), информацией о ресурсах и определить стоимость проекта.
Для этого заполнить ресурсный лист в программе \textit{MS Project}, принимая во внимание, что к реализации проекта привлекается не более 10 исполнителей.
Предусмотреть, что стандартная ставка ресурса составляет 200 руб./день.
Произвести назначение ресурсов на задачи в соответствии с таблицей.
С учетом того, что квалификация ресурсов одинаковая, при назначении ресурсов использовать процент загрузки.
Для выполнения работ С и Е предусмотреть назначение материального ресурса стоимость 100 рублей за штуку и расходом 2 штуки для работы С и 5 штук для работы Е.
Таблица ресурсов:
\begin{table}[h]
	\centering
	\begin{tabularx}{\textwidth}{|X|X|}
		\hline
		\textit{Название работы} & \textit{Количество исполнителей (чел.)} \\ \hline
		Работа A        & 5                 \\ \hline
		Работа B        & 7                  \\ \hline
		Работа C        & 1                 \\ \hline
		Работа D        & 3                  \\ \hline
		Работа E        & 2                  \\ \hline
		Работа F        & 3                  \\ \hline
		Работа G        & 6                 \\ \hline
		Работа H        & 1                 \\ \hline
		Работа I        & 5                  \\ \hline
	\end{tabularx}
\end{table}

Тип задач по умолчанию~---~\textit{Фиксированный объем ресурсов}.
Так как чем больше привлекается ресурсов, тем больше вероятность ускорить выполнение.

\subsection{Результаты выполнения задачи}

На рисунке~\ref{img:test1} представлено добавление ресурса исполнителя в проект.
\includeimage
	{test1}
	{f}
	{H}
	{\textwidth}
	{Добавление ресурса исполнителя в проект}

На рисунке~\ref{img:test2} представлен список задач с добавленными исполнителями.
\includeimage
	{test2}
	{f}
	{H}
	{\textwidth}
	{Список задач с добавленными исполнителями}

На рисунке~\ref{img:test3} представлен список задач с добавленными ресурсами и стоимостью проекта.
\includeimage
	{test3}
	{f}
	{H}
	{\textwidth}
	{Список задач с добавленными ресурсами и стоимостью проекта}

\subsection{Вывод}

В результате проектирования тестового программного продукта была произведена оценка стоимости проекта~---~34.300~р.
Также по ресурс разработчиков оказался перегружен.

\clearpage
\section{Основное задание}

\subsection{Содержание проекта}

Команда разработчиков из 16 человек занимается созданием карты города на основе собственного модуля отображения. 
Проект должен быть завершен в течение 6 месяцев. 
Бюджет проекта: 50 000 рублей.

\subsection{Задание1: Создание списка ресурсов}

На рисунке~\ref{img:task1} представлен заполненный ресурсный лист.
\includeimage
	{task1}
	{f}
	{H}
	{\textwidth}
	{Сведения о проекте}

\subsection{Задание 2: Назначение ресурсов задачам}

На рисунке~\ref{img:task2} представлен список задач с назначенными ресурсами.
\includeimage
	{task2}
	{f}
	{H}
	{\textwidth}
	{Список задач с назначенными ресурсами}

\clearpage
На рисунке~\ref{img:task2_2} представлен список задач со следующими изменениями:
\begin{enumerate}
	\item задачам 2, 8 и 12 заданы по 1000 р. фиксированных затрат;
	\item для задачи № 8 «Построение базы объектов» дополнительный арендован сервер (стоимость аренды - 2 руб/ч).
\end{enumerate}
\includeimage
	{task2_2}
	{f}
	{H}
	{\textwidth}
	{Список задач с изменениями для задач 2, 8 и 12}


На рисунке~\ref{img:task2_3} представлена информация о перегрузках.
\includeimage
	{task2_3}
	{f}
	{H}
	{\textwidth}
	{Информация о перегрузках}

Таким образом, перегрузка наблюдается у системного аналитика, художника-дизайнера и технического писателя.

\subsection{Задание 3: Анализ затрат по группам ресурсов}

На рисунке~\ref{img:task3} представлена структуризация затрат по группам ресурсов.
\includeimage
	{task3}
	{f}
	{H}
	{\textwidth}
	{Структуризация затрат по группам ресурсов}

На рисунке~\ref{img:task3_1} представлена структуризация трудозатрат по группам ресурсов.
\includeimage
	{task3_1}
	{f}
	{H}
	{\textwidth}
	{Структуризация трудозатрат по группам ресурсов}

\clearpage
На рисунке~\ref{img:task3_2} представлена информация о затратах и трудозатратах по структурным группам ресурсов представьте в графическом виде.
\includeimage
	{task3_2}
	{f}
	{H}
	{\textwidth}
	{Информация о затратах и трудозатратах по структурным группам ресурсов представьте в графическом виде}

Исходя из полученных диаграмм, можно сделать следующие выводы:
\begin{enumerate}
	\item затраты на <<Программирование>> занимают большую часть (50\%);
	\item самые низкие затраты и трудозатраты имеют группы <<М-медиа>> и <<Документация>>;
	\item самая продуктивная группа - <<Ввод данных>>, так как затраты в 2.5 меньше трудозатрат;
	\item самая непродуктивная группа - <<Анализ>>, так как затраты в 5 раз больше трудозатрат;
	\item затраты и трудозатраты для групп <<М-Медиа>>, <<Документация>> и <<Internet>> имеют одинаковые процентные показатели;
\end{enumerate}
